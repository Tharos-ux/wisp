\documentclass[12pt]{article}
\usepackage[a4paper, total={6in, 8in}]{geometry}
\usepackage[utf8]{inputenc}
\usepackage[clean]{svg}
\usepackage{graphicx}
\usepackage{caption}
\usepackage{float}
\usepackage{flafter}
\usepackage{hyperref}
\title{JOB : small\_MinION\_collection\_read\_2\\[0.2em]\smaller{}READ ID : CP087590.1}
\date{2022-06-15}
\begin{document}
\maketitle
\begin{abstract}
\begin{sloppypar}
Sample has been splitted in 10000 distinct lectures over 3636985 sequenced nucleotides.\\
Explored hypothesis are all above 10 percent of attributed reads.\\
All explorations have been made within a significance range of [0, 0.1[.\\
This report was produced with WISP version 0.0.1+43.g8d0d4fc.dirty. \\
You may get source code from \url{https://github.com/Tharos-ux/wisp}
\end{sloppypar}
\end{abstract}\begin{figure}[h]
\centering
\includesvg[width=0.4\textwidth]{small\_MinION\_collection\_read\_2\_pie\_merge.svg}
\includesvg[width=0.4\textwidth]{small\_MinION\_collection\_read\_2\_tree.svg}
\caption{Global data for small\_MinION\_collection\_read\_2}
\label{f-Global data for small\_MinION\_collection\_read\_2}
\end{figure}
\begin{table}[htp]
\begin{tabular}{ll}
window\_size & 10000 \\
sampling\_objective & 500 \\
domain\_ref & [5, 50, [1, 1, 1, 1, 1]] \\
phylum\_ref & [5, 100, [1, 1, 1, 1, 1]] \\
group\_ref & [4, 100, [1, 1, 1, 1]] \\
order\_ref & [4, 100, [1, 1, 1, 1]] \\
family\_ref & [4, 100, [1, 1, 1, 1]] \\
merged\_ref & [4, 50, [1, 1, 1, 1]] \\
domain\_sample & [5, [1, 1, 1, 1, 1]] \\
phylum\_sample & [5, [1, 1, 1, 1, 1]] \\
group\_sample & [4, [1, 1, 1, 1]] \\
order\_sample & [4, [1, 1, 1, 1]] \\
family\_sample & [4, [1, 1, 1, 1]] \\
merged\_sample & [4, [1, 1, 1, 1]] \\
input\_train & genomes/train\_small/ \\
input\_unk & genomes/unk\_small/ \\
database\_output & data/ \\
reports\_output & output/small/ \\
threshold & 0.1 \\
nb\_boosts & 10 \\
tree\_depth & 10 \\
test\_mode & min\_set \\
reads\_th & 0.1 \\
selection\_mode & delta\_mean \\
force\_model\_rebuild & True \\
single\_way & True \\
targeted\_level & family \\
levels\_list & ['domain', 'phylum', 'group', 'order', 'family'] \\
abundance\_threshold & 0.25 \\
email & siegfried.dubois@inria.fr \\
annotate\_path & genomes/to\_annotate \\
accession\_numbers & genomes/assembly\_summary.txt \\
db\_name & small \\
prefix\_job & small \\
log\_file & LOG\_wisp\_small
\end{tabular}
\caption*{Algorithm parameters}
\end{table}
\clearpage
\pagebreak[4]
\section*{Level domain}
\begin{figure}[h]
\centering
\includesvg[width=0.8\textwidth]{domain\_None\_confusion\_matrix.svg}
\includesvg[width=0.8\textwidth]{domain\_None\_graph\_reads.svg}
\caption{Level domain}
\label{f-Level domain}
\end{figure}
\begin{table}[htp]
\begin{tabular}{l}
accuracy : 1.0 \\
macro avg : {'precision': 1.0, 'recall': 1.0, 'f1-score': 1.0, 'support': 96} \\
weighted avg : {'precision': 1.0, 'recall': 1.0, 'f1-score': 1.0, 'support': 96} \\
Bacteria : {'precision': 1.0, 'recall': 1.0, 'f1-score': 1.0, 'support': 96}
\end{tabular}
\caption*{Estimators for None}
\end{table}




\clearpage
\pagebreak[4]
\section*{Level phylum for hypothesis Bacteria}
\begin{figure}[h]
\centering
\includesvg[width=0.8\textwidth]{phylum\_Bacteria\_confusion\_matrix.svg}
\includesvg[width=0.8\textwidth]{phylum\_Bacteria\_graph\_reads.svg}
\caption{Level phylum for hypothesis Bacteria}
\label{f-Level phylum for hypothesis Bacteria}
\end{figure}
\begin{table}[htp]
\begin{tabular}{l}
accuracy : 1.0 \\
macro avg : {'precision': 1.0, 'recall': 1.0, 'f1-score': 1.0, 'support': 198} \\
weighted avg : {'precision': 1.0, 'recall': 1.0, 'f1-score': 1.0, 'support': 198} \\
Actinobacteria : {'precision': 1.0, 'recall': 1.0, 'f1-score': 1.0, 'support': 132} \\
Firmicutes : {'precision': 1.0, 'recall': 1.0, 'f1-score': 1.0, 'support': 66}
\end{tabular}
\caption*{Estimators for Bacteria}
\end{table}




\clearpage
\pagebreak[4]
\section*{Level group for hypothesis Firmicutes}
\begin{figure}[h]
\centering
\includesvg[width=0.8\textwidth]{group\_Firmicutes\_confusion\_matrix.svg}
\includesvg[width=0.8\textwidth]{group\_Firmicutes\_graph\_reads.svg}
\caption{Level group for hypothesis Firmicutes}
\label{f-Level group for hypothesis Firmicutes}
\end{figure}
\begin{table}[htp]
\begin{tabular}{l}
accuracy : 1.0 \\
macro avg : {'precision': 1.0, 'recall': 1.0, 'f1-score': 1.0, 'support': 66} \\
weighted avg : {'precision': 1.0, 'recall': 1.0, 'f1-score': 1.0, 'support': 66} \\
Bacilli : {'precision': 1.0, 'recall': 1.0, 'f1-score': 1.0, 'support': 66}
\end{tabular}
\caption*{Estimators for Firmicutes}
\end{table}




\clearpage
\pagebreak[4]
\section*{Level order for hypothesis Bacilli}
\begin{figure}[h]
\centering
\includesvg[width=0.8\textwidth]{order\_Bacilli\_confusion\_matrix.svg}
\includesvg[width=0.8\textwidth]{order\_Bacilli\_graph\_reads.svg}
\caption{Level order for hypothesis Bacilli}
\label{f-Level order for hypothesis Bacilli}
\end{figure}
\begin{table}[htp]
\begin{tabular}{l}
accuracy : 1.0 \\
macro avg : {'precision': 1.0, 'recall': 1.0, 'f1-score': 1.0, 'support': 66} \\
weighted avg : {'precision': 1.0, 'recall': 1.0, 'f1-score': 1.0, 'support': 66} \\
Lactobacillales : {'precision': 1.0, 'recall': 1.0, 'f1-score': 1.0, 'support': 66}
\end{tabular}
\caption*{Estimators for Bacilli}
\end{table}




\clearpage
\pagebreak[4]
\section*{Level family for hypothesis Lactobacillales}
\begin{figure}[h]
\centering
\includesvg[width=0.8\textwidth]{family\_Lactobacillales\_confusion\_matrix.svg}
\includesvg[width=0.8\textwidth]{family\_Lactobacillales\_graph\_reads.svg}
\caption{Level family for hypothesis Lactobacillales}
\label{f-Level family for hypothesis Lactobacillales}
\end{figure}
\begin{table}[htp]
\begin{tabular}{l}
accuracy : 0.9848484848484849 \\
macro avg : {'precision': 0.99, 'recall': 0.98, 'f1-score': 0.98, 'support': 66} \\
weighted avg : {'precision': 0.99, 'recall': 0.98, 'f1-score': 0.98, 'support': 66} \\
Lactiplantibacillus : {'precision': 1.0, 'recall': 0.97, 'f1-score': 0.98, 'support': 33} \\
Lactococcus : {'precision': 0.97, 'recall': 1.0, 'f1-score': 0.99, 'support': 33}
\end{tabular}
\caption*{Estimators for Lactobacillales}
\end{table}




\end{document}